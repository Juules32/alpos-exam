\problemname{Benny's Biggest Band}

In the small town of New York Benny is appalled by the lack of young people attending the live performances of the apex of American musical culture - jazz. Though this tragedy is of course very discouraging he has thought up a plan to remedy the situation: establish a big band. Like a true American he thinks that if he can make his big band big enough it must surely be better and attract even the youngest of audiences. And, like a true performer, he has done some research on his target audience and concluded that they want a boomin' sound. What does a boomin' big band sound like? Benny imagines it means that the entire range of audible sounds is balanced i.e. that there is an equal number of instruments in each range and one musician playing an instrument cannot cover more than one range.

But just how big of a boomin' band can Benny actually make?

\section*{Input}

The first line consists of three integers $1 \le m \le 1000, 1 \le i \le 100, 1 \le r \le 10$: the number of musicians, instruments, and ranges respectively.

Then follow m lines containing the name of a musician and the instruments they play.

Lastly, are i lines containing the name of an instrument, how many Benny has in stock ($ \le 1000$), and the ranges it can play in. 

\section*{Output}

A line containing a single integer representing the maximum amount of musicians Benny can use while keeping the ranges balanced.
